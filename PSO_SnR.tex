\documentclass[journal]{IEEEtran}
\usepackage[utf8]{inputenc}
\usepackage[spanish]{babel}
\usepackage{amsmath}
\usepackage{amsfonts}
\usepackage{amssymb}
\usepackage{graphicx}
\usepackage{float}
\author{Cesar Mauricio Forero Cardozo}
\title{Algoritmo PSO aplicado a Busqueda y Rescate con Drones en Zonas de Gran Superficie }
\begin{document}
\maketitle
\begin{abstract}
El presente articulo busca brindar una solución eficiente y efectiva para el apoyo en labores de busqueda y rescate en lugares de dificil recorrido. Esto haciendo uso de un ``Enjambre de Drones'' controlados mediante algoritmos comportamentales de PSO que resulten en una alternatia mucho más economica que los protocolos de rescate actualmente usados.\\

Así pues el principal objetivo es optimizar los gastos economicos que las instituciones deberían de emplear para llevar a cabo rescates de personas perdidas en cualquier tipo de terreno, ademas de disminuir tambien el tiempo de busqueda mediante esfuerzos combinados de hombre-maquina aumentando así las probabilidades de supervivencia de las personas perdidas.
\end{abstract}
\section{Introducción}
Según el ministerio de ambiente de Colombia, la cantidad creciente de personas (residentes y extranjeras) que acuden a los parques nacionales del país desde el 2015 ha causado, como es de suponer, un aumento proporcional en la cantidad de personas que se extravían en los territorios de los mismos; por esta razón el Minambiente ha obligado a los turistas a adquirir polizas de seguro que cubran los gastos de los protocolos establecidos para busqueda y rescate.\\

Los costos de los rescates varian de acuerdo a la complejidad geografica, la cantidad de personas perdidas, las condiciones climatologicas y varios factores más que hacen imposibe estimar un presupuesto previo al rescate. Estos factores tambien afectan directamente el tiempo que se deba emplear en la busqueda de los individuos.\\

Actualmente se emplean metodologias de busqueda especializadas que mezclan componentes empiricos y teoricos con el fin de optimizar el rendimiento de los protocolos de busqueda y rescate, sin embargo se esta comenzando a implementar nuevas tecnologias que aceleren el proceso, como por ejemplo uso de naves no tripuladas, simulaciones computacionales e incluso seguimiento satelital.\\

\section{Metodos tipicos de Busqueda y Rescate}
\subsection{Caracterización}
El primer paso para llevar a cabo un rescate efectivo es la caracterización de el personal que se extravio, así como la zona en la que sucedio, esto con el fin de abstraer cierta información crucial para comenzar la busqueda y que resultarán en delimitaciones y priorizaciones de las areas de busqueda.\\

Uno de los criterios más importantes a tener en cuenta es la edad de la persona extraviada, esto debido a que este rasgo en particular nos puede dar una referencia comportamental y psicologicas en general pudiendo deducir las probabilidades de encontrarles en ciertos tipos de lugares.\\

Para los niños entre 4 y 6 años (ver Fig. 1) que se encuentran en su etapa preoperatoria, ya entienden las relaciones de causa y efecto, por lo que su tendencia es la de ir por sennderos y cominos, sin embargo de llegar a alterarse pueden desorientarse y alejarse de dichos puntos .\\

\begin{figure}[H]
\begin{center}
\includegraphics[width=0.35\textwidth]{4a6}
\label{fig:Figure 01}
\textbf{\caption{``Probabilidad de localización niños entre 4 a 6 Años''}}
\end{center}
\end{figure}

Para los niños entre 7 a 12 años (ver Fig. 2) dada su mayor capacidad de orientación, les resulta más facil conservar la calma, orientarse y volver de regreso al punto de partida (ULC).\\

\begin{figure}[H]
\begin{center}
\includegraphics[width=0.35\textwidth]{7a12}
\label{fig:Figure 02}
\textbf{\caption{``Probabilidad de localización niños entre 7 a 12 Años''}}
\end{center}
\end{figure}

En la edad entre 13 a 17 años (ver Fig. 3) el niño tendera a moverse a zonas en donde pueda adquirir mayor información para tomar decisiones a cerca de donde moverse, p.e. lugares altos o poblados.\\

\begin{figure}[H]
\begin{center}
\includegraphics[width=0.35\textwidth]{13a17}
\label{fig:Figure 03}
\textbf{\caption{``Probabilidad de localización niños entre 13 a 17 Años''}}
\end{center}
\end{figure}

Para un adulto (ver Fig. 4), debido a que la causa más probable de perdida sea la de tomar un camino por error, la confianza en su propia cognición y su incapacidad de admitir equivocaciones hacen que el individuo tienda a alejarse mas del ULC.\\

\begin{figure}[H]
\begin{center}
\includegraphics[width=0.35\textwidth]{18oMas}
\label{fig:Figure 04}
\textbf{\caption{``Probabilidad de localización adultos mayores a 18 Años''}}
\end{center}
\end{figure}

\subsection{ULC y radio de busqueda}
Según los protocolos actuales de busqueda y rescate, el primer acercamiento que se debe realizar tras la carecterización es el establecimiento de el "Ultimo Lugar Conicido" (ULC) de la persona perdida; a partir de este y con criterios preestablecidos según la condición física y psicologica del individuo, se determina un radio a traves del que la persona pudo haberse desplazado en linea recta (ver Fig. 5), en este punto estos acercamientos se realizan independientemente de la geologia del terreno.


\begin{figure}[H]
\begin{center}
\includegraphics[width=0.35\textwidth]{Radio_SnR}
\label{fig:Figure 5}
\textbf{\caption{``Radio de busqueda determinadas desde el ULC''}}
\end{center}
\end{figure}

Vale la pena resaltar que este método es completamente empírico dependiente en gran manera de la experticia que posea la cabeza del equipo de busqueda y rescate; además las áreas que se obtienen aumentan de manera exponencial conforme pasa el tiempo, dificultando la busqueda de una manera gradual.

\subsection{Busquedas con Drones en la Actualidad}
Actualmente la busqueda de grandes áreas con drones se lleva a cabo de manera mecanica y predefinida; los drones patrullan en la manera que se dispongan en el programa, de manera paralela con camaras de calor o de alta definición y enviando las señales captadas a la central para un análisis manual.\\

Este mécanismo (la busqueda con naves no tripuladas) más que usarse como medio principal, se usa como una metodología de apoyo para ir descartando zonas de patrullaje de manera paulatina pero sin una supervisión automatizada.\\

Al contrario de los excesivos costos de un rescate con vehiculos aereos tripulados, que se encuentran sobre los 4000 euros o 15 millones de pesos colombianos (tan solo el despliegue), la comodidad de transporte y despegue de los drones ademas de funcionar con batería los hacen extremadamente economicos para este proposito, y por ende una opción tentativa muy viable. 

\section{PSO en busqueda y rescate}

\subsection{Algoritmo PSO}

Optimización por Enjambre de Particulas o PSO por sus siglas en ingles (\textit{Particle Swamp Optimization}), es un algoritmo que se basa en el movimiento de las particulas sobre un espacio y su comunicación para encontrar óptimos locales y compararlo con los encontrados por otros componentes del enjambre para así llegar a una notable aproximación del óptimo global.\\

Su funcionamiento consiste basicamente en individuos que iran recorriendo el espacio (``Particula'') ponderando de manera constante el valor de el lugar especifico en el que se encuentran, tras esto dicha ``Particula'' le comunica a todas las otras (``Enjambre'') el valor más alto que ha encontrado hasta el momento, en el caso de que el ``Enjambre'' haya encontrado un valor más alto (y dependiendo de la prioridad que la ``Particula'' tenga del optimo encontrado por el ``Enjambre'') la ``Particula'' tendera a moverse hacia el óptimo encontrado por el ``Enjambre''.\\

El algoritmo PSO posee variables que son ajustables para poder obtener el comportamiento del enjambre deseado:\\

\begin{itemize}
\item \textbf{Cantidad de particulas:} Como es de esperarse, una gran cantidad de particulas aumentan la probabilidad de llegar al óptimo globar de manera más rápida.
\item \textbf{Inercia de la particula:} Al ser las particulas simulaciones de objetos que poseen masa, tambien deberán poseer una inercia subyacente.
\item \textbf{Prioridad del óptimo local:} La prioridad que la Particula le da al óptimo que el mismo encontro, sobre el óptimo encontrado por el enjambre.
\item \textbf{Prioridad del óptimo global:} La prioridad que la Particula le da al óptimo encontrado por el enjambre sobre el óptimo que el mismo encontro.
\end{itemize}

\subsection{Normalización de PSO para uso en Drones}

Dada la gran cantidad de aplicaciones que el algoritmo PSO tiene en una gran variedad de ambitos, es de esperar que se deban hacer ciertos ajustes sobre el mismo para que funcione sobre el nicho en el cual se va aplicar, a esto le llamaremos normalización del algoritmo.\\

Uno de los principales ajustes que debemos realizar es la manera en la que el algorimo realiza la ponderación de la región en la que la Particula (en adelante Drone) se encuentra, y que posteriormete comunicará al enjambre de Drones.\\

Para solucionar esto, deberemos contemplar el hecho de que los Drones iran equipados con camaras termicas y de alta resolución (Actualmente ya lo están) que permitirá mediante un algortimo básico de análisis de imagen, establecer un porcentaje de probabilistico que indique que tan factible es que la imagen captada corresponda a el objetivo buscado.\\

Así pues la ponderación se realizará de cero a uno, y será dado por cada uno de los Drones que sobrevuelen dicha región.\\

Por otro lado los Drones deben poseer configuraciones de velocidad ya que algunas orografiás del terreno permiten recorrerse de manera más sencilla que otras más complejas, sin embargo se debe tener en cuenta que este aspecto sumado a la inercia del Drone puede generar comportamientos no deseados, es por esto que la velocidad máxima del Drone se deja a criterio del usuario.\\
 
El punto de partida tambien es un aspecto importante de nuestro algoritmo, ya que normalmente cada Particula tiene un punto aleatorio de partida, con lo que su cobertura del espacio es mayor, sin embargo eso en nuestras condiciones es irreal, por lo que se decidio colocar un punto estatico de origen referente al ULC y asignar direcciones y velocidades aleatorias (que no sobrepasen lo establecido por el usuario) a cada uno de los componentes del enjambre de Drones.\\

Por último, es muy común que un objeto pueda pasar como un acierto sobre el objetivo que se está buscando, a esto le consideramos "Falso Positivo", este se deberá confirmar de manera manual enviando un equipo de rescate.\\
 
\section{NetLogo como Herramienta para Simulación}

Netlogo es un lenguaje de programación basado en agentes que dado su paradigma de programación facilita de gran manera la utilización de grupos, colonias, enjambres, etc... de individuos que tengan reglas básicas de comportamiento y así poder simular comportamientos mucho más complejos que resultan de la emergencia de los lazos comportamentales y las interaciones entre los mismos.\\
 
Dada esta gran ventaja brindada por el lenguaje de programación, es una herramienta excelente para realizar implementaciones de algoritmos bioinspirados como el flocking, ABC o en el algoritmo que nos ocupa PSO.\\

\subsection{Generalidades de NetLogo}
Netlogo utiliza una terminología bastante peculiar, teniendo un espacio de nombres, expresiones regulares y palabras reservadas propias del lenguaje:\\

\begin{itemize}
\item \textbf{Tortuga/Turtle: } Cada Tortuga o \textit{Turtle} es un agente que hace parte de un grupo mas grande del mismo, así pues para nuestro caso cada \textit{Turtle} se referirá a un Drone.
\item \textbf{Parcela/Patch: } Cada Parcela o \textit{Patch} es un trozo de la pantalla de simulación y representa una región con la que interactuán todos los \textit{Turtles}.
\item \textbf{Set Up: } El bloque de código de \textit{Set Up} es la configuración inicial que se realiza antes de cada ejecución.
\item \textbf{GO: } El bloque de código de \textit{Go} es la ejecución que se realizará de forma iterativa  para todas o solo algunas de las \textit{Turtles}.
\item \textbf{Ticks: } Cada \textit{Tick} representa una Iteración en la que se lleva a cabo todo el bloque de código escrito dentro de la instrucción \textbf{\textit{GO}}. 
\end{itemize}

\subsection{Simulación en NetLogo}
Para realizar la simulación del algoritmo PSO normalizado para busqueda y rescate, se dispusierón de tres paneles que ofrecen una diversidad de información al usuario final.\\

En el panel de configuración (ver Fig. 6) podemos modificar los valores preestablecidos, para obtener otro tipo de comportamiento (grupal o individual) de los Drones, también podemos cambiar la dificultad de recorrido de el terreno y decidir si el óptimo global encontrado por el algoritmo es o no un "Falso Positivo".\\

\begin{figure}[H]
\begin{center}
\includegraphics[width=0.4\textwidth]{Panel_Configuracion}
\label{fig:Figure 6}
\textbf{\caption{``Panel de Configuración''}}
\end{center}
\end{figure}

En el panel de Graficas (ver Fig. 8) podemos observar comportamientos grupales como la cantidad de Drones que estan en el objetivo especificado a lo largo del tiempo y el costo del rescato en ese mismo intervalo.\\

\begin{figure}[H]
\begin{center}
\includegraphics[width=0.4\textwidth]{Panel_Simulacion}
\label{fig:Figure 7}
\textbf{\caption{``Panel de Simulación de PSO''}}
\end{center}
\end{figure}

En el panel de  simulación (ver Fig. 7) podemos ver el comportamiento de los Drones siendo guiados por el algoritmo de PSO en tiempo de ejecución, allí podemos ver también el reflejo de la orografía del terreno elejida por el usuario además de los falsos positivos que se hayan generado a lo largo de la ejecución.\\

\begin{figure}[H]
\begin{center}
\includegraphics[width=0.4\textwidth]{Panel_Graficas}
\label{fig:Figure 8}
\textbf{\caption{``Panel de Información Gráfica''}}
\end{center}
\end{figure}

En cuanto al código a ejecutar, como se comento en el apartado anterior se divide en dos grandes bloques que se pueden sintentizar en el \textit{Set Up} (ver Fig. 9) en donde tenemos la configuración inicial de los Drones y sus caracteristicas, p.e. posición, velocidad, dirección, etc... Por otro lado en el bloque de código \textit{Go} (ver Fig. 10) encontramos la evaluación individual de las Parcelas en las que el Drone se encuentra y la comparativa con los valores encontradas a nivel grupal, posteriormente la toma de decisión de el proximo movimiento a realizar.\\

\begin{figure}[H]
\begin{center}
\includegraphics[width=0.4\textwidth]{FlujoSetup}
\label{fig:Figure 9}
\textbf{\caption{``Diagrama de fujo de el código en SetUp''}}
\end{center}
\end{figure}

\begin{figure}[H]
\begin{center}
\includegraphics[width=0.4\textwidth]{FlujoGo}
\label{fig:Figure 10}
\textbf{\caption{``Diagrama de fujo de el código en Go''}}
\end{center}
\end{figure}

\section{Resultados}
Tras una serie de ejecuciones en las que se evaluó el comportamiento tanto de manera individual como de manera grupal, se establecieron unos valores predefinidos que describen el comportamiento deseado para el enjambre de Drones, sin embargo y como ya se dijo estos valoes se pueden modificar en cualquier momento por el usuario final desde el Panel de Configuración o incluso directamente desde el código del bloque de \textit{Set Up}.\\

\begin{itemize}
\item $Inercia del Drone = 0.98\frac{m}{s}$
\item $Max velocidad del Drone = 50\frac{m}{s} $
\item $Prioridad al Max Personal = 0.5 $
\item $Prioridad al Max Grupal = 1.2 $ \\

\end{itemize}

Tambien se estableció que en promedio (ver Fig. 11) se deben enviar 7 ayudas, es decir se obtienen en promedio 6 Falsos positivos, cada vez que se realiza una busqueda en un campo de gran superficie, también se esteblece que el tiempo que se debe emplear en el rescate es de unas 10 horas con un costo promedio de 7800 USD (Principalmente debido al desplazamiento de equipos de rescate tras los falsos positivos).\\

\begin{figure}[H]
\begin{center}
\includegraphics[width=0.4\textwidth]{Tabla}
\label{fig:Figure 11}
\textbf{\caption{``Tabla de Promedios en Serie de Ejecuciones''}}
\end{center}
\end{figure}

\section{Conclusiones}
Dado que un rescate normal por helicoptero tiene unos costos elevadisimos (al rededor de 5480 USD solo el despegue del helicoptero) el ahorrar este tipo de recursos para rescates unicamente de alta complejidad se hace necesario, es por esto que el uso de algoritmos automatizados de localización de objetivos en grandes superficies es tan factible; por otro lado el tiempo que se debe emplear en promedio para la localización y posterior rescate es notablemente menor al que se suele emplear en rescates de este tipo (al rededor de 1 o 2 días) por lo que de llegar a pulirse el algoritmo normalizado, contemplando ciertos criterios que por cuestiones de complejidad se obviaron (p.e. el clima) resulta mucho más rentable que el protocolo de resacte actual.\\

\section{Referencias}
[1] Minambiente. (2018,Dic,09). ``Póliza de accidentes y rescate en Parques Nacionales'' [Online] . Available: http://www.parquesnacionales.gov.co/portal/es/poliza-de-accidentes-y-rescate-en-parques-nacionales/\\

[2] P.H. Eugenio, \textit{``Rescate y Búsqueda en Grandes Áreas,''} in Manual de Rescate y Salvamento Vol. 7. Guadalajara: Griker Orgemer, 2015.\\

[3] Fernando Caparrini. (2016,Nov,18). ``PSO: Optimización por enjambres de partículas'' [Online]. Available: http://www.cs.us.es/~fsancho/?e=70.\\

[4] N. Rokbani, A. Abraham and A. M. Alimi, "Fuzzy Ant Supervised by PSO and simplified ant supervised PSO applied to TSP," 13th International Conference on Hybrid Intelligent Systems (HIS 2013), Gammarth, 2013, pp. 251-255.\\

[5]D. Pal, P. Verma, D. Gautam and P. Indait, "Improved optimization technique using hybrid ACO-PSO," 2016 2nd International Conference on Next Generation Computing Technologies (NGCT), Dehradun, 2016, pp. 277-282.\\


[6] Servidrone. ``Preguntas Frecuentes (Drones)'' [Online]. Available: https://www.servidrone.com/faq/\\

[7] Emilio Ordiz. (2018,Oct,29). ``Las dudas en los rescates: quién los paga, cuánto cuestan y en qué situaciones se acude'' [Online]. Available: https://www.20minutos.es/noticia/3230755/0/preguntas-respuestas-rescates-emergencias/\\

[8] Li Wei and Liu wenyuan, "The optimized selection methods of marine search and rescue ships," 2016 13th International Conference on Service Systems and Service Management (ICSSSM), Kunming, 2016, pp. 1-4.\\

[9]H. Yang and W. Chiu, "Cooperative multirobot strategy for seeking multiple sources," 2017 International Conference on Applied System Innovation (ICASI), Sapporo, 2017, pp. 1779-1781.\\

[10]D. Kurabayashi, H. Asama, K. Noda, I. Endo and H. Hashimoto, "Information assistance in rescue using intelligent data carriers," Proceedings 2001 IEEE/RSJ International Conference on Intelligent Robots and Systems. Expanding the Societal Role of Robotics in the the Next Millennium, Maui, HI, USA, 2001, pp. 2294-2299 vol.4

\end{document}